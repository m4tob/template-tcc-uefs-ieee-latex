\documentclass[conference]{IEEEtran}

\usepackage[utf8x]{inputenc}
\usepackage[T1]{fontenc}
\usepackage{graphicx}
% \usepackage[colorlinks=true, allcolors=blue]{hyperref}
\usepackage[brazil]{babel}
\usepackage{algorithm}
\usepackage[noend]{algpseudocode}
\usepackage{amsmath,amssymb,amsfonts}
\usepackage[numbers]{natbib}
\usepackage{multirow}
\usepackage{caption}
\usepackage{subcaption}

\usepackage{url}
\def\UrlBreaks{\do\/\do-}
\usepackage{breakurl}

\hyphenation{op-tical net-works semi-conduc-tor}

\begin{document}

\title{Título do Trabalho}
\author{\IEEEauthorblockN{Nome Completo do Autor}
    \IEEEauthorblockA{
        \textit{Universidade Estadual de Feira de Santana}\\
        Feira de Santana, Brasil\\
        email@uefs.br
    }
    \and
    \IEEEauthorblockN{Nome Completo do Orientador}
    \IEEEauthorblockA{
        \textit{Universidade Estadual de Feira de Santana}\\
        Feira de Santana, Brasil\\
        email@uefs.br
    }
}

\maketitle

\thispagestyle{plain}
\pagestyle{plain}

\begin{abstract}
Escrever um texto que contemple todo o conteúdo do trabalho. Constituído de parágrafo único; uma sequência de frases concisas e objetivas e não de uma simples enumeração de tópicos, não ultrapassando 500 palavras, O resumo deve ressaltar o objetivo, o método, os resultados e as conclusões do documento
\end{abstract}

\renewcommand\IEEEkeywordsname{Palavras-chave}
\begin{IEEEkeywords}
palavra 1, palavra 2, palavra 3, palavra 4....
\end{IEEEkeywords}

\IEEEpeerreviewmaketitle

\section{Introdução}
\footnotetext[1]{Notas de rodapé. Podem ser usados para informar copyright ou coisas do tipo.}

Apresentar a temática no contexto mais amplo e, em seguida, o contexto mais específico. Justificar a necessidade/importância da pesquisa para o estado da arte e para área. Apresentar os objetivos da pesquisa. Apresentar estrutura do texto. \cite{IEEEexample:techrepstdsub}. Quando necessário, usar notas de rodapé\footnotemark[1]

\section{Fundamentação Teórica}
Descreva os conceitos centrais do seu TCC, a partir de livros, artigos e periódicos. É recomendável que se faça um levantamento de trabalhos relacionados ao seu, destacando as principais contribuições.

Respeitar a autoria, nas citações diretas e indiretas. Evitar parágrafos muito longos. Evitar seções e subseções muito curtas. Você pode referenciar figuras com Figura \ref{fig:figura}, equações com Equação \ref{eq:equacao} e/ou tabelas com Tabela \ref{tabela}

\begin{figure}[htbp]
    \centerline{\includegraphics{images/figure.png}}
    \caption{Exemplo de uma figura com legenda. Fonte: Adaptada de \cite{gonzales2009}}
    \label{fig:figura}
\end{figure}

\begin{figure*}[htbp]
    \centering
    \begin{subfigure}[]{0.329\textwidth}
        \centerline{ \includegraphics[width=0.8\textwidth]{images/figure_a.png} }
        \caption{Figura A}
        \label{fig:figura_2a}
    \end{subfigure}
    \hfill
    \begin{subfigure}[]{0.329\textwidth}
        \centerline{ \includegraphics[width=0.8\textwidth]{images/figure_b.png} }
        \caption{Figura B}
        \label{fig:figura_2b}
    \end{subfigure}
    \hfill
    \begin{subfigure}[]{0.329\textwidth}
        \centerline{ \includegraphics[width=0.8\textwidth]{images/figure_c.png} }
        \caption{Figura C}
        \label{fig:figura_2c}
    \end{subfigure}
    \caption{Exemplo de uma figura com subfiguras e ocupando as duas colunas do documento}
    \label{fig:figura_ocupando_duas_colunas}
\end{figure*}

\begin{align} \label{eq:equacao}
    acur\acute{a}cia = \frac{VP + VN}{total}
\end{align}

\begin{table}[htbp]
    \begin{center}
        \renewcommand{\arraystretch}{1.4}
        \caption{Exemplo de tabela com legenda. Fonte: Próprio Autor.}
        \begin{tabular}{*{4}{c}}%
            \hline
            Acurácia(\%)   & Precisão(\%) & Revocação(\%)    & \textit{F\textsubscript{1}-Score}(\%) \\ \hline
            94,64           & 96,79         & 89,98             & 93,26         \\
            \label{tab:tabela}
        \end{tabular}
    \end{center}
\end{table}

\section{Metodologia}\label{sec::pspotterk}
Parte do texto onde são apresentados os materiais e métodos utilizados para o desenvolvimento do trabalho. As descrições apresentadas devem ser suficientes para permitir a compreensão das etapas da pesquisa, no entanto, minúcias de provas matemáticas ou procedimentos experimentais, se necessários, devem constituir material anexo.

\section{Resultados}
Parte do texto onde são apresentados os resultados obtidos com o trabalho e a discussão dos mesmos.

\section{Conclusão}
Parte do texto que contem as conclusões correspondentes aos objetivos 

\ifCLASSOPTIONcaptionsoff
  \newpage
\fi

\bibliographystyle{IEEEtran}
\bibliography{IEEEabrv.bib,contents/referencias.bib}

\end{document}